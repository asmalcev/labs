\documentclass[a4paper,14pt]{extarticle}
\usepackage[utf8]{inputenc}
\usepackage[english, russian]{babel}
\usepackage{graphicx}
\usepackage[top=0.8in, bottom=0.8in, left=0.8in, right=0.8in]{geometry}
\usepackage{pgfplots}
\usepackage{amsmath}
\usepackage{setspace}
\usepackage{titlesec}
\usepackage{float}
\usepackage{chngcntr}
\usepackage{pgfplots}
\usepackage{amsfonts}
\usepackage{pgfplotstable}
\usepackage{multirow}
\usepackage{karnaugh-map}
\usepackage{tikz,xcolor}
\usepackage{listings}

\titleformat{\section}[hang]
  {\bfseries}
  {}
  {0em}
  {\hspace{-0.4pt}\large \thesection\hspace{0.6em}}
  
  
\titleformat{\subsection}[hang]
  {\bfseries}
  {}
  {0em}
  {\hspace{-0.4pt}\large \thesubsection\hspace{0.6em}}

%\linespread{1.3} % полуторный интервал
%\renewcommand{\rmdefault}{ftm} % Times New Roman

\newcommand{\nx}{\overline{x}}
\newcommand{\p}{0.31}
\newcommand{\scale}{1.4}

\counterwithin{figure}{section}
\counterwithin{equation}{section}
\counterwithin{table}{section}

\lstdefinestyle{CStyle}{
    basicstyle=\footnotesize,
    breakatwhitespace=false,         
    breaklines=true,                  
    keepspaces=true,                 
    numbers=left,                    
    numbersep=5pt,                  
    showspaces=false,                
    showstringspaces=false,
    showtabs=false,                  
    tabsize=2,
    language=C
}
\lstdefinestyle{Output}{
    basicstyle=\footnotesize,
    breakatwhitespace=false,         
    breaklines=true,                                    
    keepspaces=true,                                                     
    showspaces=false,                
    showstringspaces=false,
    showtabs=false,                  
    tabsize=2
}

\begin{document}
\begin{titlepage}
\centering
\small Балтийский государственный технический университет «Военмех» им. Д.Ф.Устинова \\
\vspace{3cm}
\normalsize Кафедра И5\\
«Информационные системы и программная инженерия»\\
\vspace{3cm}
\textbf{Практическое задание №3}\\
по дисциплине Основы программирования на тему\\ 
\textbf{«Указатели»}\\
\vfill

\begin{flushleft}
\textbf{Выполнил:}
\hfill {Мальцев А.С.} \\
\hfill {Группа И595} \\
\vspace{1cm}
\textbf{Преподаватель:}
\hfill {Лазарева Т.И.} \\
\end{flushleft}
\vspace{3cm}

{\centering Санкт-Петербург \\ 
\vspace{0.15cm}
2019}
\end{titlepage}

\section{Цель работы}
Цель работы - ознакомиться с адресацией памяти, научиться правильно использовать указатели различных типов.

\section{Ход работы}
\subsection{Задание 1}
Набрать текст программы, представленный ниже. Проанализировать выдаваемые программой результаты. Объяснить, почему они именно такие.
\lstinputlisting[language=c, frame=single, label=lab2_2, style=CStyle]{../firstTask.c}
\begin{center}
Текст программы
\end{center}
a:      int: start address 0x7ffc90c8f078   extent 4 \\
// выводится ссылка на ячейку памяти, где лежит значение перменной a, под переменную этого типа(int) отводится 4 байта\\
b:    float: start address 0x7ffc90c8f07c   extent 4\\
// выводится ссылка на ячейку памяти, где лежит значение перменной b, под переменную этого типа(float) отводится 4 байта\\
c:   double: start address 0x7ffc90c8f080   extent 8\\
// выводится ссылка на ячейку памяти, где лежит значение перменной c, под переменную этого типа(double) отводится 8 байт\\
\\
p1: pointer: start address 0x7ffc90c8f088   extent 8\\
// выводится ссылка на ячейку памяти, где лежит значение указателя p1, под указатель отводится 8 байт(т.к. система 64-битная)\\
p2: pointer: start address 0x7ffc90c8f090   extent 8\\
// выводится ссылка на ячейку памяти, где лежит значение указателя p2, под указатель отводится 8 байт(т.к. система 64-битная)\\
p3: pointer: start address 0x7ffc90c8f098   extent 8\\
// выводится ссылка на ячейку памяти, где лежит значение указателя p3, под указатель отводится 8 байт(т.к. система 64-битная)\\
\\
p1: 0x7ffc90c8f078 related value 1\\
// выводится значение указателя p1 - ссылка на значение переменной a, с помощью оператора разыменования получают значение a = 1\\
p2: 0x7ffc90c8f07c related value 2.000000\\
// выводится значение указателя p2 - ссылка на значение переменной b, с помощью оператора разыменования получают значение b = 2.000000\\
p3: 0x7ffc90c8f080 related value 3.000000\\
// выводится значение указателя p1 - ссылка на значение переменной c, с помощью оператора разыменования получают значение c = 3.000000\\
\\
a=1	b=2.000000	c=3.000000\\
// выводятся значения переменных a,b,c\\
a=5	b=10.000000	c=1.732051\\
// в переменную, на которую указывает p1, записали 5, в переменную, на которую указывает p2, записали произведение значений переменных, на которые указывают p2 и p1, в переменную, на которую указывает p3, записали квадратный корень из прошлого значения\\
*p1=5	*p2=10.000000	*p3=1.732051\\
// с помощью оператора разыменования были выведены значения переменных, на которые указывают p1, p2 и p3\\

p1=0x7ffc90c8f07c	p2=0x7ffc90c8f07c	p3=0x7ffc90c8f07c	p4=0x7ffc90c8f07c\\
// в указатель p1 записывают значение указателя p2 с явным приведением его значения к int, p2 без изменений, в указатель p3 записывают значение указателя p2 с явным приведением его значения к double, в указатель p4 записывают значение указателя p2\\
*p1=1092616192	*p2=10.000000	*p3=-44345960814313427366288584061119198\\6994302798471837256132976627811013080170989229007181138033542832573601\\2158457591225030600146976299635938600943270244460674065232929344128657\\06388667269788991488.000000	*(float*)p4=  10.000000\\
// с помощью оператора разыменования выводятся значения, на которые указывают p1, p2, p3, p4. они все указывают на одно и то же значение, но являются указателями разных типов, поэтому выводятся разные значения. при выводе p4 используется явное приведение к float, для корректного отображения\\
\\
p1=0x7ffc90c8f080	p2=0x7ffc90c8f07c	p3=0x7ffc90c8f074\\
// в случае с p1 и p3 произошел сдвиг указателя на ячейки памяти, которые находятся на количество байт, выделяемое под запись числа их типа(у int и float - 4, у double - 8), вправо и влево соответственно\\
*p1=-396866390		*p2=10.000000	*p3=0.000000\\
// с помощью оператора разыменования выведены значения передвинутых указателей\\
p1=0x7ffc90c8f070	p2=0x7ffc90c8f07c	p3=0x7ffc90c8f070\\
// произошел сдвиг указателей, выведены их новые ссылки. p1 сместили на 16 байт влево, p3 присвоили ссылку на переменную a c явным приведение типа к double, а потом вычли 1\\
*p1=-1865879336	*p2=10.000000	*p3=0.000000\\
// выведены значения обновленных указателей p1, p2, p3\\
\begin{center}
Вывод программы с объяснениями
\end{center}



\subsection{Задание 2}
Набрать текст программы, найти в нем синтаксические ошибки и исправить их, в начало добавить вывод на экран адресов всех переменных, а в конец -- значение всех переменных, проанализировать полученные результаты и объяснить, почему они именно такие. Заменить оператор m += 2, оператором m++, проанализировать результат.
\lstinputlisting[language=c, frame=single, label=lab2_2, style=CStyle]{../secondTask.c}
\begin{center}
Текст программы
\end{center}
\&p = 0x7ffd91d52a30, \&c = 0x7ffd91d52a27,\\ \&a = 0x7ffd91d52a38, \&b = 0x7ffd91d52a28,\\
\&x = 0x7ffd91d52a40, \&y = 0x7ffd91d52a2c,\\ \&m = 0x7ffd91d52a48, \&n = 0x7ffd91d52a50\\\\
Enter b =10\\
a=0x7ffd91d52a28	*a=10	b=10\\
// указатель а содержит ссылку на переменную b, с помощью оператора разыменования подтверждается то, что значение, на которое указывает а, равно значению b\\ 
p=0x7ffd91d52a28	c=10	a=0x7ffd91d52a28	b=167772160\\
// в указатель р передают значение указателя а, а переменной с присваивают значение, на которое указывают р и а
x=0x7ffd91d52a2c	*x=3.500000	y=3.500000\\
a=0x7ffd91d52a2c	*a=3	x=0x7ffd91d52a2c	*x=0.000000	y=0.000000\\
x=0x7ffd91d52a2c	*x=12345.000000	y=12345.000000\\
p=0x7ffd91d52a2c	c=0	x=0x7ffd91d52a2c	y=0.000000\\
m=0x7ffd91d52a50	*m=0.000000	n=0.000000\\
m=0x7ffd91d52a50	*m=5.500000	n=5.500000\\
b=1	y=1.700000	n=1.700000\\
*a=1	*x=1.700000	*m=1.700000\\
n=1.700000	n=0x7ffd91d52a50	m=0x7ffd91d52a60\\
m=0x7ffd91d52a60	*m=0.300000\\
p = 0x7ffd91d52a2c, c = 0, a = 0x7ffd91d52a28, b = 1,\\
x = 0x7ffd91d52a2c, y = 1.700000, m = 0x7ffd91d52a60, n = 1.700000\\

\begin{center}
Вывод программы
\end{center}

\end{document}
