\documentclass[a4paper,14pt]{extarticle}
\usepackage[utf8]{inputenc}
\usepackage[russian]{babel}
\usepackage{graphicx}
\usepackage[top=0.8in, bottom=0.8in, left=0.8in, right=0.8in]{geometry}
\usepackage{pgfplots}
\usepackage{amsmath}
\usepackage{setspace}
\usepackage{titlesec}
\usepackage{float}
\usepackage{chngcntr}
\usepackage{pgfplots}
\usepackage{amsfonts}
\usepackage{pgfplotstable}
\usepackage{multirow}
\usepackage{karnaugh-map}
\usepackage{tikz,xcolor}
\usepackage{listings}

\titleformat{\section}[hang]
  {\bfseries}
  {}
  {0em}
  {\hspace{-0.4pt}\large \thesection\hspace{0.6em}}
  
  
\titleformat{\subsection}[hang]
  {\bfseries}
  {}
  {0em}
  {\hspace{-0.4pt}\large \thesubsection\hspace{0.6em}}

\newcommand{\nx}{\overline{x}}
\newcommand{\p}{0.31}
\newcommand{\scale}{1.4}

\counterwithin{figure}{section}
\counterwithin{equation}{section}
\counterwithin{table}{section}

\lstdefinestyle{CStyle}{
    basicstyle=\footnotesize,
    breakatwhitespace=false,         
    breaklines=true,                 
    captionpos=b,                    
    keepspaces=true,                 
    numbers=left,                    
    numbersep=5pt,                  
    showspaces=false,                
    showstringspaces=false,
    showtabs=false,                  
    tabsize=2,
    language=C
}
\lstdefinestyle{output}{
    basicstyle=\footnotesize,
    breakatwhitespace=false,         
    breaklines=true,                 
    captionpos=b,                    
    keepspaces=true
}

\begin{document}
\begin{titlepage}
\centering
\small Балтийский государственный технический университет «Военмех» им. Д.Ф.Устинова \\
\vspace{3cm}
\normalsize Кафедра И5\\
«Информационные системы и программная инженерия»\\
\vspace{3cm}
\textbf{Практическое задание №1}\\
по дисциплине Основы программирования на тему\\ 
\textbf{«Структура программы, основные типы данных, ввод/вывод»}\\
\vfill

\begin{flushleft}
\textbf{Выполнил:}
\hfill {Мальцев А.С.} \\
\hfill {Группа И595} \\
\vspace{1cm}
\textbf{Преподаватель:}
\hfill {Лазарева Т.И.} \\
\end{flushleft}
\vspace{3cm}

{\centering Санкт-Петербург \\ 
\vspace{0.15cm}
2019}
\end{titlepage}

\section{Цель работы}
Изучить структуру программы, научиться использовать переменные различных типов, освоить функции форматного ввода и вывода, арифметические операции и операции присваивания.

\section{Ход работы}
\subsection{Задача 1} 
Набрать текст программы, представленный ниже. Проанализировать значения переменных после каждой операции присваивания. Проверить порядок выполнения операций в каждом выражении, содержащем несколько операций присваивания, разделив каждый оператор-выражение на несколько операторов, выполняемых последовательно. В функциях ввода и вывода изменить спецификаторы формата, проанализировать полученные результаты.
\lstinputlisting[language=c, frame=single, caption=Программа первого задания, label=lab2_2, style=CStyle]{../first.c}
\lstinputlisting[language=c, frame=single, caption=Измененая программа первого задания, label=lab2_2, style=CStyle]{../firstTask.c}
В измененной программе каждая операция по присваиванию нового значения переменной выполняется в отдельной строке.

\subsection{Задача 2} 
Написать программу для вычисления значений следующих выражений:\\
a = 5, c = 5\\
a = a + b - 2\\
c = c + 1, d = c - a + d\\
a = a  * c, c = c - 1\\
a = a / 10, c = c / 2, b = b - 1, d = d * (c + b + a)\\
Выражения, записанные в одной строке, записывать одним оператором-выражением. Переменные c и d объявить как целые, переменные a и b – как вещественные. Значения переменных b и d вводить с клавиатуры. После вычисления каждого выражения выводить на экран значения всех переменных.
\lstinputlisting[language=c, frame=single, caption=Программа второго задания, label=lab2_2, style=CStyle]{../secondTask.c}
\begin{center}
\begin{tabular}{|c|cccc|}
\hline
Исходные данные & \multicolumn{4}{|c|}{Вывод программы}\\
\hline
b = 4 & 5.000  & 4.000 & 5 & 8 \\
d = 8 & 7.000  & 4.000 & 5 & 8 \\
& 7.000  & 4.000 & 6 & 7 \\
& 42.000 & 4.000 & 5 & 7 \\
& 4.200  & 3.000 & 2 & 64 \\
\hline
\end{tabular}\\
\end{center}
При входных данных 4 и 8, начальные значения переменных: a = 5, b = 4,\\ c = 5, d = 8. На строке 9: а = 5 + 4 - 2 = 7. На 11 строке: c = 6,\\ d = 8 + 6 - 7 = 7. 13 строка: a = 7 * 6 = 42, c = 5. 15 строка: c = 2, b = 3, a = 4.2, d = 7 * (4.2 + 2 + 3) = 64.4 = 64 т.к. d - переменная целочисленного типа.

\subsection{Задача 3} 
Набрать текст программы, представленный ниже. Проанализировать выдаваемые программой результаты. Объяснить, почему они именно такие.
\lstinputlisting[language=c, frame=single, caption=Программа третьего задания, label=lab2_2, style=CStyle]{../thirdTask.c}
\lstinputlisting[language=c, frame=single, caption=Вывод программы, label=lab2_2, style=output]{../output}
На 16 строке выводятся длины переменных в байтах. 
На 18 строке выводится максимальное значение типа signed char, записанное в две переменные. На 20 строке вывод -128 и 128, т.к. диапазон возможных значений для типа signed char равен [-128, 127], поэтому происходит переполнение разрядной сетки; диапазон unsigned char - [0, 255], поэтому число 128 записалось в переменную uc. 
Строка 22: в обе переменные записано минимальное значение типа signed char --- -128, но в unsigned char нет отрицательных значений, поэтому эта двоичная запись числа считывается как 128. 
Строка 24: максимальное значение unsigned char --- 255, но оно не входит в диапазон signed char, поэтому считывается как -1. 
Строка 26: исходя из прошлых выводов можно сделать вывод, что signed char считывает как -1 + 1 = 0, а в случае с unsigned char происходит переполнение разрядной сетки, поэтому становится равно 0. 
Строка 28: точно как и в предыдущих примерах unsigned char не может работать с отрицательными значениями, а signed char может. 
Строка 30: значения, хранящиеся в с и uc , т.е. -5 и 5 соответственно сравниваются как обычные целочисленные значения, поэтому -5 > 5 --- не верно. 
Строка 33: максимальным возможным значением short int является число 32767, оно не укладывается в диапазон unsigned и signed типов char, поэтому они принимают значения -1 и 255. 
Строка 35: диапазон возможных значений short int --- [-32768, 32767], 32768 переполняет разрядную сетку этого типа данных, поэтому значение переменной становится -32768. 
Строка 37: значение -32768 не входит в допустимый диапазон значений для типов данных signed char и unsigned char, поэтому принимают значение 0. 
Строка 39: в переменные s, c и uc записывается 0. 
Строка 42: в переменные l, i и u записывают максимальное значение для типа int - 2147483647. 
Строка 44: диапазон значений типа long int позволяет записать большие значения, в данном случае 2147483648, в случае с unsigned int диапазон возможных значений не включает отрицательные числа, поэтому остается больше памяти для положительных, а именно [0, 4294967295], поэтому в переменную u так же записывается число 2147483648, но ситуация с int другая --- происходит переполнение разрядной сетки, поэтому значение становится -2147483648. 
Строка 47: в переменные i, l и u записывают минимальное значение типа int --- -2147483648, но unsigned int не работает с отрицательными значениями, поэтому устанавливается значение 2147483648. 
Строка 50: в переменные  i, l и u записывается максимальное значение типа unsigned int --- 4294967295, в u и l получилось записать, а в случае с i происходит переполнение разрядной сетки, поэтому устанавливается значение -1. 
Строка 52: в переменные i и u записывают значение -5, но unsigned char не работает с отрицательными числами, поэтому в u сохраняется значение 5. 
Строка 54:  для сравнения двух разных типов данных, int пиводится к unsigned int, и становится равным 4294967291 > 5, поэтому условное выражение возвращает 1. 
Строка 56: по аналогии с предыдущим пунктом, выполняется приведение типов для их сравнения, поэтому вывод так же 1. 
Строка 58: в переменной целочисленного типа не может хранится вещественное число, поэтому дробная часть попросту отбрасывается. 
Строка 60: так же как в прошлом выводе. 
Строка 62: в переменные типов float и double записывается максимальное значение для типа float, диапазон значений типа double гораздо шире, чем float. 
Строка 64: так же, как в прошлом выводе, только с минимальным значением float. 
Строка 66: в переменные f и d записали число, меньше которого невозможно задавать относительную точность вещественных чисел типа float. 
Строка 68: в f записали $10^{10}$. 
Строка 70: в f записали $10^{11}$, но произошла потеря точности. 
Строка 72: по аналогии с прошлым выводом, произошла потеря точности. 
Строка 74: в переменную d записали максимальное значение типа double. 
Строка 76: в переменную d записали минимальное значение типа double. 
Строка 78: в переменную d записали число, меньше которого невозможно задавать относительную точность вещественных чисел типа double. 
Строка 80: в d записали $10^{15}+1$. 
Строка 82: в d записали $10^{16}$ + 1, но произошла потеря точности. 
Строка 86: при записи большого числа в переменную f, т.к. операции происходят последовательно, каждая в своей строке, число 1 теряется, поэтому выводится 0. 
Строка 88: выполены те же вычисления, что и в предыдущем выводе, но теперь все выражение размещено в одной строке, поэтому сначала находится значение выражения, а лишь потом записывается в перменную f. 
Строка 90: значение выражение записали в переменную d. 
Строка 92: потеря точности происходит из-за слишком больших чисел, поэтому число 1000 теряется.

\end{document}
