\documentclass[a4paper,14pt]{extarticle}
\usepackage[utf8]{inputenc}
\usepackage[russian]{babel}
\usepackage{graphicx}
\usepackage[top=0.8in, bottom=0.8in, left=0.8in, right=0.8in]{geometry}
\usepackage{pgfplots}
\usepackage{amsmath}
\usepackage{setspace}
\usepackage{titlesec}
\usepackage{float}
\usepackage{chngcntr}
\usepackage{pgfplots}
\usepackage{amsfonts}
\usepackage{pgfplotstable}
\usepackage{multirow}
\usepackage{karnaugh-map}
\usepackage{tikz,xcolor}
\usepackage{listings}

\titleformat{\section}[hang]
  {\bfseries}
  {}
  {0em}
  {\hspace{-0.4pt}\large \thesection\hspace{0.6em}}
  
  
\titleformat{\subsection}[hang]
  {\bfseries}
  {}
  {0em}
  {\hspace{-0.4pt}\large \thesubsection\hspace{0.6em}}

%\linespread{1.3} % полуторный интервал
%\renewcommand{\rmdefault}{ftm} % Times New Roman

\newcommand{\nx}{\overline{x}}
\newcommand{\p}{0.31}
\newcommand{\scale}{1.4}

\counterwithin{figure}{section}
\counterwithin{equation}{section}
\counterwithin{table}{section}

\lstdefinestyle{CStyle}{
    basicstyle=\footnotesize,
    breakatwhitespace=false,         
    breaklines=true,                 
    captionpos=b,                    
    keepspaces=true,                 
    numbers=left,                    
    numbersep=5pt,                  
    showspaces=false,                
    showstringspaces=false,
    showtabs=false,                  
    tabsize=2,
    language=C
}

\begin{document}
\begin{titlepage}
\centering
\small Балтийский государственный технический университет «Военмех» им. Д.Ф.Устинова \\
\vspace{3cm}
\normalsize Кафедра И5\\
«Информационные системы и программная инженерия»\\
\vspace{3cm}
\textbf{Практическое задание №2}\\
по дисциплине Основы программирования на тему\\ 
\textbf{«Ветвления и циклы»}\\
\vfill

\begin{flushleft}
\textbf{Выполнил:}
\hfill {Мальцев А.С.} \\
\hfill {Группа И595} \\
\vspace{1cm}
\textbf{Преподаватель:}
\hfill {Лазарева Т.И.} \\
\end{flushleft}
\vspace{3cm}

{\centering Санкт-Петербург \\ 
\vspace{0.15cm}
2019}
\end{titlepage}

\section{Цель работы}
Познакомиться с функциями из математической библиотеки, освоить операции отношения и логические операции, научиться грамотно программировать базовые алгоритмические структуры «ветвление» и «цикл».

\section{Ход работы}
\begin{center}
{\large\bf Вариант 14}
\end{center}
\subsection{Задание 1}
В программе вычисляется значение функции, с использованием тернарного оператора. 
\lstinputlisting[language=c, frame=single, caption=Программа первого задания, label=lab2_2, style=CStyle]{../firstTask.c}

\subsection{Задание 2}
В программе вычисляется значение по указанной формуле, с использованием функции математической библиотеки.
\lstinputlisting[language=c, frame=single, caption=Программа второго задания, label=lab2_2, style=CStyle]{../secondTask.c}

\subsection{Задание 3}
Программа, написанная с помощью вложенного условного оператора.
\lstinputlisting[language=c, frame=single, caption=Программа третьего задания, label=lab2_2, style=CStyle]{../thirdTask.c}

\subsection{Задание 4}
В программе смоделирован арифметический цикл с помощью оператора цикла for. 
\lstinputlisting[language=c, frame=single, caption=Программа четвертого задания, label=lab2_2, style=CStyle]{../fourthTask.c}


\end{document}
