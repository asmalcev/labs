\documentclass[a4paper,14pt]{extarticle}
\usepackage[utf8]{inputenc}
\usepackage[russian]{babel}
\usepackage{graphicx}
\usepackage[top=0.8in, bottom=0.8in, left=0.8in, right=0.8in]{geometry}
\usepackage{pgfplots}
\usepackage{amsmath}
\usepackage{setspace}
\usepackage{titlesec}
\usepackage{float}
\usepackage{chngcntr}
\usepackage{pgfplots}
\usepackage{amsfonts}
\usepackage{pgfplotstable}
\usepackage{multirow}
\usepackage{karnaugh-map}
\usepackage{tikz,xcolor}
\usepackage{listings}

\titleformat{\section}[hang]
  {\bfseries}
  {}
  {0em}
  {\hspace{-0.4pt}\large \thesection\hspace{0.6em}}
  
  
\titleformat{\subsection}[hang]
  {\bfseries}
  {}
  {0em}
  {\hspace{-0.4pt}\large \thesubsection\hspace{0.6em}}

\newcommand{\nx}{\overline{x}}
\newcommand{\p}{0.31}
\newcommand{\scale}{1.4}

\counterwithin{figure}{section}
\counterwithin{equation}{section}
\counterwithin{table}{section}

\lstdefinestyle{CStyle}{
  basicstyle=\footnotesize,
  breakatwhitespace=false,         
  breaklines=true,                 
  captionpos=b,                    
  keepspaces=true,                 
  numbers=left,                    
  numbersep=5pt,                  
  showspaces=false,                
  showstringspaces=false,
  showtabs=false,                  
  tabsize=2,
  language=C
}
\lstdefinestyle{FStyle}{
  basicstyle=\footnotesize,
  breakatwhitespace=false,         
  breaklines=true,                 
  keepspaces=true,                 
  showspaces=false,                
  showstringspaces=false,
  showtabs=false,                  
  tabsize=2
}

\begin{document}
\begin{titlepage}
\centering
\small Балтийский государственный технический университет «Военмех» им. Д.Ф.Устинова \\
\vspace{3cm}
\normalsize Кафедра И5\\
«Информационные системы и программная инженерия»\\
\vspace{3cm}
\textbf{Практическое задание №7}\\
по дисциплине Основы программирования на тему\\ 
\textbf{«Файлы»}\\
\begin{center}
{\large\bf Вариант 14}
\end{center}
\vfill

\begin{flushleft}
\textbf{Выполнил:}
\hfill {Мальцев А.С.} \\
\hfill {Группа И595} \\
\vspace{1cm}
\textbf{Преподаватель:}
\hfill {Лазарева Т.И.} \\
\end{flushleft}
\vspace{3cm}

{\centering Санкт-Петербург \\ 
\vspace{0.15cm}
2019}
\end{titlepage}
\setcounter{page}{2}
\section{Цель работы}
Ознакомиться с потоковыми функциями языка С для работы с текстовыми и бинарными файлами, научиться использовать прямой и последовательный доступ к данным, хранящимся в файлах.

\section{Ход работы}
\subsection{Вспомогательный файл с функциями}
\lstinputlisting[language=c, frame=single, style=CStyle]{../functions.c}
\begin{center}
Текст программы\\
\vspace{0.6cm}
\end{center}

\subsection{Задание 1}
Дан файл, содержащий некоторый текст. Оставить в этом файле только те фразы, в которых имеются слова, записанные прописными буквами.\\
\textit{Исходные файл:} файл с фразами.\\
\textit{Результирующий файл:} новый файл на основании исходного.\\
\lstinputlisting[language=c, frame=single, style=CStyle]{../1.c}
\begin{center}
  Текст программы
\end{center}
\vspace{0.6cm}
\subsubsection{Исходный файл}
\lstinputlisting[language=c, frame=single, style=FStyle]{../input2}
\subsubsection{Результат}
\lstinputlisting[language=c, frame=single, style=FStyle]{../output2}
\vspace{0.3cm}
\begin{center}
  Результаты тестирования
\end{center}

\subsection{Задание 2}
Дан текстовый файл, содержащий целые числа. Увеличить значения четных чисел вдвое, остальные оставить без изменения.\\
\textit{Исходные файл:} файл с целыми числами.\\
\textit{Результирующий файл:} новый файл на основании исходного.\\
\lstinputlisting[language=c, frame=single, style=CStyle]{../2.c}
\begin{center}
Текст программы
\end{center}
\vspace{0.6cm}
\subsubsection{Исходный файл}
\lstinputlisting[language=c, frame=single, style=FStyle]{../input2}
\subsubsection{Результат}
\lstinputlisting[language=c, frame=single, style=FStyle]{../output2}
\vspace{0.3cm}
\begin{center}
  Результаты тестирования
\end{center}



\subsection{Задание 3}
Компоненты бинарного файла – вещественные числа. Поменять местами первый и последний отрицательные компоненты. В конец файла добавить количество отрицательных компонентов.\\
\textit{Исходные файл:} файл с вещественными числами.\\
\textit{Результирующий файл:} новый файл на основании исходного.\\
\lstinputlisting[language=c, frame=single, style=CStyle]{../3.c}
\begin{center}
Текст программы
\end{center}
\vspace{0.3cm}
\lstinputlisting[language=c, frame=single, style=CStyle]{../binaryGenerator.c}
\begin{center}
Генератор бинарного файла
\end{center}
\vspace{0.6cm}
\subsubsection{Исходный файл}
\lstinputlisting[language=c, frame=single, style=FStyle]{../input.bin}
\subsubsection{Результат}
\lstinputlisting[language=c, frame=single, style=FStyle]{../output.bin}
\vspace{0.3cm}
\begin{center}
  Результаты тестирования
\end{center}

\subsection{Задание 4}
В файле хранятся сведения о вкладчиках банка: номер счета, паспортные данные, категория вклада, текущая сумма вклада, дата последней операции. Зафиксировать (произвести изменения) операции приема и выдачи любой суммы. Вывести наибольшую сумму вклада в категории «срочный».\\
\textit{Исходные файл:} файл с сведениями о вкладчиках.\\
\textit{Результирующий файл:} измененый исходный файл.\\
\lstinputlisting[language=c, frame=single, style=CStyle]{../4.c}
\begin{center}
Текст программы\\
\vspace{0.6cm}
\end{center}

\end{document}
